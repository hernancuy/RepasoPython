\documentclass[]{article}
\usepackage{amsmath}
\usepackage{amsthm}
\usepackage[utf8]{inputenc}
\usepackage{listings}

%\usepackage{amsfonts}
%\usepackage{amssymb}
%\usepackage{graphicx}

\begin{document}
	
	
\title{Esqueleto para verificación de programas con ciclos en latex}
	
\author{Jorge Duitama}
	
\maketitle

\section*{Programa a verificar}

Escribir el programa, función o procedimiento en el siguiente entorno lstlisting.
Utilzar el símbolo \$ para incluir símbolos matemáticos

\begin{lstlisting}[language=GCL, mathescape, tabsize=4]


\end{lstlisting}

\section*{Paso 1}
$\{Q\}INIC\{P\}$

\begin{proof}

Si INIC es un programa complejo, se recomienda comenzar escribiendo la tripla a probar

\begin{lstlisting}[language=GCL, mathescape, tabsize=4]
{$Q$}
INIC
{$P$}
\end{lstlisting}

Y realizar la prueba a continuación

\begin{align*}
 & \\
\equiv &<\text{}> \\
 & \\
\equiv &<\text{}> \\
 & \\
\equiv &<\text{}> \\
 & 
\end{align*}
\end{proof}

\section*{Paso 2}
$P \wedge \neg BC \implies R$

\begin{proof}
\begin{align*}
& \\
\equiv &<\text{}> \\
& \\
\equiv &<\text{}> \\
& \\
\equiv &<\text{}> \\
&
\end{align*}
\end{proof}


\section*{Paso 3}
$\{P \wedge BC\}SC\{P\}$ 
\begin{proof}

Se recomienda comenzar escribiendo la tripla a probar

\begin{lstlisting}[language=GCL, mathescape, tabsize=4]
{$P \wedge BC$}
SC
{$P$}
\end{lstlisting}

Y realizar la prueba a continuación

\begin{align*}
& \\
\equiv &<\text{}> \\
& \\
\equiv &<\text{}> \\
& \\
\equiv &<\text{}> \\
& 
\end{align*}

Si la prueba es compleja se puede hacer en varias partes

\begin{align*}
& \\
\equiv &<\text{}> \\
& \\
\equiv &<\text{}> \\
& \\
\equiv &<\text{}> \\
&
\end{align*}

\end{proof}

\section*{Paso 4}
$P \wedge BC \implies t>0$
\begin{proof}


Se debe definir la función de cota t

\begin{align*}
& \\
\equiv &<\text{}> \\
& \\
\equiv &<\text{}> \\
& \\
\equiv &<\text{}> \\
&
\end{align*}
\end{proof}

\section*{Paso 5}
$\{P \wedge BC \wedge t=C\}SC\{t<C\}$
\begin{proof}

Se recomienda comenzar escribiendo la tripla a probar, de acuerdo con la cota definida en el paso anterior

\begin{lstlisting}[language=GCL, mathescape, tabsize=4]
{$P \wedge BC \wedge t=C$}
SC
{$t<C$}
\end{lstlisting}

Y realizar la prueba a continuación

\begin{align*}
& \\
\equiv &<\text{}> \\
& \\
\equiv &<\text{}> \\
& \\
\equiv &<\text{}> \\
& 
\end{align*}

\end{proof}
	
\end{document}