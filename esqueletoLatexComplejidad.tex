\documentclass[]{article}
\usepackage{amsmath}
\usepackage{amsthm}
\usepackage[utf8]{inputenc}
\usepackage{listings}

\begin{document}
	
	
\title{Esqueleto en LaTeX para análisis de complejidad}
	
\author{Jorge Duitama}
	
\maketitle

\section*{Programa a analizar}

Escribir el programa, función o procedimiento en el siguiente entorno lstlisting.
Utilzar el símbolo \$ para incluir símbolos matemáticos

\begin{lstlisting}[language=GCL, mathescape, tabsize=4]


\end{lstlisting}

\section*{Derivación de fórmula}
La siguiente tabla resume las operaciones que se ejecutan y la cantidad de veces que se ejecuta cada una:

\begin{table}[h!]
\begin{center}
\begin{tabular}{|l|l|l|}
\hline
\textbf{Operación(es)} & \textbf{Constante} & \textbf{Veces que se ejecuta} \\
\hline
Asignación ($:=$) & $c_1$ & $ $ \\
Suma, Resta ($+,-$) & $c_2$ & $ $ \\
Mult., Div., Módulo  ($*,/,mod$) & $c_3$ & $ $ \\
Comparación ($<,>,\leq,\geq,=,\neq$) & $c_4$ & $ $ \\
\hline
\end{tabular}
\end{center}
\end{table}

\section*{Solucion de ecuación de recurrencia}
En caso de que el análisis de complejidad llegue a una ecuación de recurrencia, resolver esta ecuación. Se puede utilizar el entorno align para organizar manipulación de ecuaciones


\begin{align*}
 &=  \\	
 &=  \\
 &= \\
 &= \\
& 
\end{align*}

\end{document}